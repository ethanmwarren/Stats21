\documentclass[11pt]{article}

    \usepackage[breakable]{tcolorbox}
    \usepackage{parskip} % Stop auto-indenting (to mimic markdown behaviour)
    
    \usepackage{iftex}
    \ifPDFTeX
    	\usepackage[T1]{fontenc}
    	\usepackage{mathpazo}
    \else
    	\usepackage{fontspec}
    \fi

    % Basic figure setup, for now with no caption control since it's done
    % automatically by Pandoc (which extracts ![](path) syntax from Markdown).
    \usepackage{graphicx}
    % Maintain compatibility with old templates. Remove in nbconvert 6.0
    \let\Oldincludegraphics\includegraphics
    % Ensure that by default, figures have no caption (until we provide a
    % proper Figure object with a Caption API and a way to capture that
    % in the conversion process - todo).
    \usepackage{caption}
    \DeclareCaptionFormat{nocaption}{}
    \captionsetup{format=nocaption,aboveskip=0pt,belowskip=0pt}

    \usepackage{float}
    \floatplacement{figure}{H} % forces figures to be placed at the correct location
    \usepackage{xcolor} % Allow colors to be defined
    \usepackage{enumerate} % Needed for markdown enumerations to work
    \usepackage{geometry} % Used to adjust the document margins
    \usepackage{amsmath} % Equations
    \usepackage{amssymb} % Equations
    \usepackage{textcomp} % defines textquotesingle
    % Hack from http://tex.stackexchange.com/a/47451/13684:
    \AtBeginDocument{%
        \def\PYZsq{\textquotesingle}% Upright quotes in Pygmentized code
    }
    \usepackage{upquote} % Upright quotes for verbatim code
    \usepackage{eurosym} % defines \euro
    \usepackage[mathletters]{ucs} % Extended unicode (utf-8) support
    \usepackage{fancyvrb} % verbatim replacement that allows latex
    \usepackage{grffile} % extends the file name processing of package graphics 
                         % to support a larger range
    \makeatletter % fix for old versions of grffile with XeLaTeX
    \@ifpackagelater{grffile}{2019/11/01}
    {
      % Do nothing on new versions
    }
    {
      \def\Gread@@xetex#1{%
        \IfFileExists{"\Gin@base".bb}%
        {\Gread@eps{\Gin@base.bb}}%
        {\Gread@@xetex@aux#1}%
      }
    }
    \makeatother
    \usepackage[Export]{adjustbox} % Used to constrain images to a maximum size
    \adjustboxset{max size={0.9\linewidth}{0.9\paperheight}}

    % The hyperref package gives us a pdf with properly built
    % internal navigation ('pdf bookmarks' for the table of contents,
    % internal cross-reference links, web links for URLs, etc.)
    \usepackage{hyperref}
    % The default LaTeX title has an obnoxious amount of whitespace. By default,
    % titling removes some of it. It also provides customization options.
    \usepackage{titling}
    \usepackage{longtable} % longtable support required by pandoc >1.10
    \usepackage{booktabs}  % table support for pandoc > 1.12.2
    \usepackage[inline]{enumitem} % IRkernel/repr support (it uses the enumerate* environment)
    \usepackage[normalem]{ulem} % ulem is needed to support strikethroughs (\sout)
                                % normalem makes italics be italics, not underlines
    \usepackage{mathrsfs}
    

    
    % Colors for the hyperref package
    \definecolor{urlcolor}{rgb}{0,.145,.698}
    \definecolor{linkcolor}{rgb}{.71,0.21,0.01}
    \definecolor{citecolor}{rgb}{.12,.54,.11}

    % ANSI colors
    \definecolor{ansi-black}{HTML}{3E424D}
    \definecolor{ansi-black-intense}{HTML}{282C36}
    \definecolor{ansi-red}{HTML}{E75C58}
    \definecolor{ansi-red-intense}{HTML}{B22B31}
    \definecolor{ansi-green}{HTML}{00A250}
    \definecolor{ansi-green-intense}{HTML}{007427}
    \definecolor{ansi-yellow}{HTML}{DDB62B}
    \definecolor{ansi-yellow-intense}{HTML}{B27D12}
    \definecolor{ansi-blue}{HTML}{208FFB}
    \definecolor{ansi-blue-intense}{HTML}{0065CA}
    \definecolor{ansi-magenta}{HTML}{D160C4}
    \definecolor{ansi-magenta-intense}{HTML}{A03196}
    \definecolor{ansi-cyan}{HTML}{60C6C8}
    \definecolor{ansi-cyan-intense}{HTML}{258F8F}
    \definecolor{ansi-white}{HTML}{C5C1B4}
    \definecolor{ansi-white-intense}{HTML}{A1A6B2}
    \definecolor{ansi-default-inverse-fg}{HTML}{FFFFFF}
    \definecolor{ansi-default-inverse-bg}{HTML}{000000}

    % common color for the border for error outputs.
    \definecolor{outerrorbackground}{HTML}{FFDFDF}

    % commands and environments needed by pandoc snippets
    % extracted from the output of `pandoc -s`
    \providecommand{\tightlist}{%
      \setlength{\itemsep}{0pt}\setlength{\parskip}{0pt}}
    \DefineVerbatimEnvironment{Highlighting}{Verbatim}{commandchars=\\\{\}}
    % Add ',fontsize=\small' for more characters per line
    \newenvironment{Shaded}{}{}
    \newcommand{\KeywordTok}[1]{\textcolor[rgb]{0.00,0.44,0.13}{\textbf{{#1}}}}
    \newcommand{\DataTypeTok}[1]{\textcolor[rgb]{0.56,0.13,0.00}{{#1}}}
    \newcommand{\DecValTok}[1]{\textcolor[rgb]{0.25,0.63,0.44}{{#1}}}
    \newcommand{\BaseNTok}[1]{\textcolor[rgb]{0.25,0.63,0.44}{{#1}}}
    \newcommand{\FloatTok}[1]{\textcolor[rgb]{0.25,0.63,0.44}{{#1}}}
    \newcommand{\CharTok}[1]{\textcolor[rgb]{0.25,0.44,0.63}{{#1}}}
    \newcommand{\StringTok}[1]{\textcolor[rgb]{0.25,0.44,0.63}{{#1}}}
    \newcommand{\CommentTok}[1]{\textcolor[rgb]{0.38,0.63,0.69}{\textit{{#1}}}}
    \newcommand{\OtherTok}[1]{\textcolor[rgb]{0.00,0.44,0.13}{{#1}}}
    \newcommand{\AlertTok}[1]{\textcolor[rgb]{1.00,0.00,0.00}{\textbf{{#1}}}}
    \newcommand{\FunctionTok}[1]{\textcolor[rgb]{0.02,0.16,0.49}{{#1}}}
    \newcommand{\RegionMarkerTok}[1]{{#1}}
    \newcommand{\ErrorTok}[1]{\textcolor[rgb]{1.00,0.00,0.00}{\textbf{{#1}}}}
    \newcommand{\NormalTok}[1]{{#1}}
    
    % Additional commands for more recent versions of Pandoc
    \newcommand{\ConstantTok}[1]{\textcolor[rgb]{0.53,0.00,0.00}{{#1}}}
    \newcommand{\SpecialCharTok}[1]{\textcolor[rgb]{0.25,0.44,0.63}{{#1}}}
    \newcommand{\VerbatimStringTok}[1]{\textcolor[rgb]{0.25,0.44,0.63}{{#1}}}
    \newcommand{\SpecialStringTok}[1]{\textcolor[rgb]{0.73,0.40,0.53}{{#1}}}
    \newcommand{\ImportTok}[1]{{#1}}
    \newcommand{\DocumentationTok}[1]{\textcolor[rgb]{0.73,0.13,0.13}{\textit{{#1}}}}
    \newcommand{\AnnotationTok}[1]{\textcolor[rgb]{0.38,0.63,0.69}{\textbf{\textit{{#1}}}}}
    \newcommand{\CommentVarTok}[1]{\textcolor[rgb]{0.38,0.63,0.69}{\textbf{\textit{{#1}}}}}
    \newcommand{\VariableTok}[1]{\textcolor[rgb]{0.10,0.09,0.49}{{#1}}}
    \newcommand{\ControlFlowTok}[1]{\textcolor[rgb]{0.00,0.44,0.13}{\textbf{{#1}}}}
    \newcommand{\OperatorTok}[1]{\textcolor[rgb]{0.40,0.40,0.40}{{#1}}}
    \newcommand{\BuiltInTok}[1]{{#1}}
    \newcommand{\ExtensionTok}[1]{{#1}}
    \newcommand{\PreprocessorTok}[1]{\textcolor[rgb]{0.74,0.48,0.00}{{#1}}}
    \newcommand{\AttributeTok}[1]{\textcolor[rgb]{0.49,0.56,0.16}{{#1}}}
    \newcommand{\InformationTok}[1]{\textcolor[rgb]{0.38,0.63,0.69}{\textbf{\textit{{#1}}}}}
    \newcommand{\WarningTok}[1]{\textcolor[rgb]{0.38,0.63,0.69}{\textbf{\textit{{#1}}}}}
    
    
    % Define a nice break command that doesn't care if a line doesn't already
    % exist.
    \def\br{\hspace*{\fill} \\* }
    % Math Jax compatibility definitions
    \def\gt{>}
    \def\lt{<}
    \let\Oldtex\TeX
    \let\Oldlatex\LaTeX
    \renewcommand{\TeX}{\textrm{\Oldtex}}
    \renewcommand{\LaTeX}{\textrm{\Oldlatex}}
    % Document parameters
    % Document title
    \title{HW3}
    
    
    
    
    
% Pygments definitions
\makeatletter
\def\PY@reset{\let\PY@it=\relax \let\PY@bf=\relax%
    \let\PY@ul=\relax \let\PY@tc=\relax%
    \let\PY@bc=\relax \let\PY@ff=\relax}
\def\PY@tok#1{\csname PY@tok@#1\endcsname}
\def\PY@toks#1+{\ifx\relax#1\empty\else%
    \PY@tok{#1}\expandafter\PY@toks\fi}
\def\PY@do#1{\PY@bc{\PY@tc{\PY@ul{%
    \PY@it{\PY@bf{\PY@ff{#1}}}}}}}
\def\PY#1#2{\PY@reset\PY@toks#1+\relax+\PY@do{#2}}

\@namedef{PY@tok@w}{\def\PY@tc##1{\textcolor[rgb]{0.73,0.73,0.73}{##1}}}
\@namedef{PY@tok@c}{\let\PY@it=\textit\def\PY@tc##1{\textcolor[rgb]{0.25,0.50,0.50}{##1}}}
\@namedef{PY@tok@cp}{\def\PY@tc##1{\textcolor[rgb]{0.74,0.48,0.00}{##1}}}
\@namedef{PY@tok@k}{\let\PY@bf=\textbf\def\PY@tc##1{\textcolor[rgb]{0.00,0.50,0.00}{##1}}}
\@namedef{PY@tok@kp}{\def\PY@tc##1{\textcolor[rgb]{0.00,0.50,0.00}{##1}}}
\@namedef{PY@tok@kt}{\def\PY@tc##1{\textcolor[rgb]{0.69,0.00,0.25}{##1}}}
\@namedef{PY@tok@o}{\def\PY@tc##1{\textcolor[rgb]{0.40,0.40,0.40}{##1}}}
\@namedef{PY@tok@ow}{\let\PY@bf=\textbf\def\PY@tc##1{\textcolor[rgb]{0.67,0.13,1.00}{##1}}}
\@namedef{PY@tok@nb}{\def\PY@tc##1{\textcolor[rgb]{0.00,0.50,0.00}{##1}}}
\@namedef{PY@tok@nf}{\def\PY@tc##1{\textcolor[rgb]{0.00,0.00,1.00}{##1}}}
\@namedef{PY@tok@nc}{\let\PY@bf=\textbf\def\PY@tc##1{\textcolor[rgb]{0.00,0.00,1.00}{##1}}}
\@namedef{PY@tok@nn}{\let\PY@bf=\textbf\def\PY@tc##1{\textcolor[rgb]{0.00,0.00,1.00}{##1}}}
\@namedef{PY@tok@ne}{\let\PY@bf=\textbf\def\PY@tc##1{\textcolor[rgb]{0.82,0.25,0.23}{##1}}}
\@namedef{PY@tok@nv}{\def\PY@tc##1{\textcolor[rgb]{0.10,0.09,0.49}{##1}}}
\@namedef{PY@tok@no}{\def\PY@tc##1{\textcolor[rgb]{0.53,0.00,0.00}{##1}}}
\@namedef{PY@tok@nl}{\def\PY@tc##1{\textcolor[rgb]{0.63,0.63,0.00}{##1}}}
\@namedef{PY@tok@ni}{\let\PY@bf=\textbf\def\PY@tc##1{\textcolor[rgb]{0.60,0.60,0.60}{##1}}}
\@namedef{PY@tok@na}{\def\PY@tc##1{\textcolor[rgb]{0.49,0.56,0.16}{##1}}}
\@namedef{PY@tok@nt}{\let\PY@bf=\textbf\def\PY@tc##1{\textcolor[rgb]{0.00,0.50,0.00}{##1}}}
\@namedef{PY@tok@nd}{\def\PY@tc##1{\textcolor[rgb]{0.67,0.13,1.00}{##1}}}
\@namedef{PY@tok@s}{\def\PY@tc##1{\textcolor[rgb]{0.73,0.13,0.13}{##1}}}
\@namedef{PY@tok@sd}{\let\PY@it=\textit\def\PY@tc##1{\textcolor[rgb]{0.73,0.13,0.13}{##1}}}
\@namedef{PY@tok@si}{\let\PY@bf=\textbf\def\PY@tc##1{\textcolor[rgb]{0.73,0.40,0.53}{##1}}}
\@namedef{PY@tok@se}{\let\PY@bf=\textbf\def\PY@tc##1{\textcolor[rgb]{0.73,0.40,0.13}{##1}}}
\@namedef{PY@tok@sr}{\def\PY@tc##1{\textcolor[rgb]{0.73,0.40,0.53}{##1}}}
\@namedef{PY@tok@ss}{\def\PY@tc##1{\textcolor[rgb]{0.10,0.09,0.49}{##1}}}
\@namedef{PY@tok@sx}{\def\PY@tc##1{\textcolor[rgb]{0.00,0.50,0.00}{##1}}}
\@namedef{PY@tok@m}{\def\PY@tc##1{\textcolor[rgb]{0.40,0.40,0.40}{##1}}}
\@namedef{PY@tok@gh}{\let\PY@bf=\textbf\def\PY@tc##1{\textcolor[rgb]{0.00,0.00,0.50}{##1}}}
\@namedef{PY@tok@gu}{\let\PY@bf=\textbf\def\PY@tc##1{\textcolor[rgb]{0.50,0.00,0.50}{##1}}}
\@namedef{PY@tok@gd}{\def\PY@tc##1{\textcolor[rgb]{0.63,0.00,0.00}{##1}}}
\@namedef{PY@tok@gi}{\def\PY@tc##1{\textcolor[rgb]{0.00,0.63,0.00}{##1}}}
\@namedef{PY@tok@gr}{\def\PY@tc##1{\textcolor[rgb]{1.00,0.00,0.00}{##1}}}
\@namedef{PY@tok@ge}{\let\PY@it=\textit}
\@namedef{PY@tok@gs}{\let\PY@bf=\textbf}
\@namedef{PY@tok@gp}{\let\PY@bf=\textbf\def\PY@tc##1{\textcolor[rgb]{0.00,0.00,0.50}{##1}}}
\@namedef{PY@tok@go}{\def\PY@tc##1{\textcolor[rgb]{0.53,0.53,0.53}{##1}}}
\@namedef{PY@tok@gt}{\def\PY@tc##1{\textcolor[rgb]{0.00,0.27,0.87}{##1}}}
\@namedef{PY@tok@err}{\def\PY@bc##1{{\setlength{\fboxsep}{\string -\fboxrule}\fcolorbox[rgb]{1.00,0.00,0.00}{1,1,1}{\strut ##1}}}}
\@namedef{PY@tok@kc}{\let\PY@bf=\textbf\def\PY@tc##1{\textcolor[rgb]{0.00,0.50,0.00}{##1}}}
\@namedef{PY@tok@kd}{\let\PY@bf=\textbf\def\PY@tc##1{\textcolor[rgb]{0.00,0.50,0.00}{##1}}}
\@namedef{PY@tok@kn}{\let\PY@bf=\textbf\def\PY@tc##1{\textcolor[rgb]{0.00,0.50,0.00}{##1}}}
\@namedef{PY@tok@kr}{\let\PY@bf=\textbf\def\PY@tc##1{\textcolor[rgb]{0.00,0.50,0.00}{##1}}}
\@namedef{PY@tok@bp}{\def\PY@tc##1{\textcolor[rgb]{0.00,0.50,0.00}{##1}}}
\@namedef{PY@tok@fm}{\def\PY@tc##1{\textcolor[rgb]{0.00,0.00,1.00}{##1}}}
\@namedef{PY@tok@vc}{\def\PY@tc##1{\textcolor[rgb]{0.10,0.09,0.49}{##1}}}
\@namedef{PY@tok@vg}{\def\PY@tc##1{\textcolor[rgb]{0.10,0.09,0.49}{##1}}}
\@namedef{PY@tok@vi}{\def\PY@tc##1{\textcolor[rgb]{0.10,0.09,0.49}{##1}}}
\@namedef{PY@tok@vm}{\def\PY@tc##1{\textcolor[rgb]{0.10,0.09,0.49}{##1}}}
\@namedef{PY@tok@sa}{\def\PY@tc##1{\textcolor[rgb]{0.73,0.13,0.13}{##1}}}
\@namedef{PY@tok@sb}{\def\PY@tc##1{\textcolor[rgb]{0.73,0.13,0.13}{##1}}}
\@namedef{PY@tok@sc}{\def\PY@tc##1{\textcolor[rgb]{0.73,0.13,0.13}{##1}}}
\@namedef{PY@tok@dl}{\def\PY@tc##1{\textcolor[rgb]{0.73,0.13,0.13}{##1}}}
\@namedef{PY@tok@s2}{\def\PY@tc##1{\textcolor[rgb]{0.73,0.13,0.13}{##1}}}
\@namedef{PY@tok@sh}{\def\PY@tc##1{\textcolor[rgb]{0.73,0.13,0.13}{##1}}}
\@namedef{PY@tok@s1}{\def\PY@tc##1{\textcolor[rgb]{0.73,0.13,0.13}{##1}}}
\@namedef{PY@tok@mb}{\def\PY@tc##1{\textcolor[rgb]{0.40,0.40,0.40}{##1}}}
\@namedef{PY@tok@mf}{\def\PY@tc##1{\textcolor[rgb]{0.40,0.40,0.40}{##1}}}
\@namedef{PY@tok@mh}{\def\PY@tc##1{\textcolor[rgb]{0.40,0.40,0.40}{##1}}}
\@namedef{PY@tok@mi}{\def\PY@tc##1{\textcolor[rgb]{0.40,0.40,0.40}{##1}}}
\@namedef{PY@tok@il}{\def\PY@tc##1{\textcolor[rgb]{0.40,0.40,0.40}{##1}}}
\@namedef{PY@tok@mo}{\def\PY@tc##1{\textcolor[rgb]{0.40,0.40,0.40}{##1}}}
\@namedef{PY@tok@ch}{\let\PY@it=\textit\def\PY@tc##1{\textcolor[rgb]{0.25,0.50,0.50}{##1}}}
\@namedef{PY@tok@cm}{\let\PY@it=\textit\def\PY@tc##1{\textcolor[rgb]{0.25,0.50,0.50}{##1}}}
\@namedef{PY@tok@cpf}{\let\PY@it=\textit\def\PY@tc##1{\textcolor[rgb]{0.25,0.50,0.50}{##1}}}
\@namedef{PY@tok@c1}{\let\PY@it=\textit\def\PY@tc##1{\textcolor[rgb]{0.25,0.50,0.50}{##1}}}
\@namedef{PY@tok@cs}{\let\PY@it=\textit\def\PY@tc##1{\textcolor[rgb]{0.25,0.50,0.50}{##1}}}

\def\PYZbs{\char`\\}
\def\PYZus{\char`\_}
\def\PYZob{\char`\{}
\def\PYZcb{\char`\}}
\def\PYZca{\char`\^}
\def\PYZam{\char`\&}
\def\PYZlt{\char`\<}
\def\PYZgt{\char`\>}
\def\PYZsh{\char`\#}
\def\PYZpc{\char`\%}
\def\PYZdl{\char`\$}
\def\PYZhy{\char`\-}
\def\PYZsq{\char`\'}
\def\PYZdq{\char`\"}
\def\PYZti{\char`\~}
% for compatibility with earlier versions
\def\PYZat{@}
\def\PYZlb{[}
\def\PYZrb{]}
\makeatother


    % For linebreaks inside Verbatim environment from package fancyvrb. 
    \makeatletter
        \newbox\Wrappedcontinuationbox 
        \newbox\Wrappedvisiblespacebox 
        \newcommand*\Wrappedvisiblespace {\textcolor{red}{\textvisiblespace}} 
        \newcommand*\Wrappedcontinuationsymbol {\textcolor{red}{\llap{\tiny$\m@th\hookrightarrow$}}} 
        \newcommand*\Wrappedcontinuationindent {3ex } 
        \newcommand*\Wrappedafterbreak {\kern\Wrappedcontinuationindent\copy\Wrappedcontinuationbox} 
        % Take advantage of the already applied Pygments mark-up to insert 
        % potential linebreaks for TeX processing. 
        %        {, <, #, %, $, ' and ": go to next line. 
        %        _, }, ^, &, >, - and ~: stay at end of broken line. 
        % Use of \textquotesingle for straight quote. 
        \newcommand*\Wrappedbreaksatspecials {% 
            \def\PYGZus{\discretionary{\char`\_}{\Wrappedafterbreak}{\char`\_}}% 
            \def\PYGZob{\discretionary{}{\Wrappedafterbreak\char`\{}{\char`\{}}% 
            \def\PYGZcb{\discretionary{\char`\}}{\Wrappedafterbreak}{\char`\}}}% 
            \def\PYGZca{\discretionary{\char`\^}{\Wrappedafterbreak}{\char`\^}}% 
            \def\PYGZam{\discretionary{\char`\&}{\Wrappedafterbreak}{\char`\&}}% 
            \def\PYGZlt{\discretionary{}{\Wrappedafterbreak\char`\<}{\char`\<}}% 
            \def\PYGZgt{\discretionary{\char`\>}{\Wrappedafterbreak}{\char`\>}}% 
            \def\PYGZsh{\discretionary{}{\Wrappedafterbreak\char`\#}{\char`\#}}% 
            \def\PYGZpc{\discretionary{}{\Wrappedafterbreak\char`\%}{\char`\%}}% 
            \def\PYGZdl{\discretionary{}{\Wrappedafterbreak\char`\$}{\char`\$}}% 
            \def\PYGZhy{\discretionary{\char`\-}{\Wrappedafterbreak}{\char`\-}}% 
            \def\PYGZsq{\discretionary{}{\Wrappedafterbreak\textquotesingle}{\textquotesingle}}% 
            \def\PYGZdq{\discretionary{}{\Wrappedafterbreak\char`\"}{\char`\"}}% 
            \def\PYGZti{\discretionary{\char`\~}{\Wrappedafterbreak}{\char`\~}}% 
        } 
        % Some characters . , ; ? ! / are not pygmentized. 
        % This macro makes them "active" and they will insert potential linebreaks 
        \newcommand*\Wrappedbreaksatpunct {% 
            \lccode`\~`\.\lowercase{\def~}{\discretionary{\hbox{\char`\.}}{\Wrappedafterbreak}{\hbox{\char`\.}}}% 
            \lccode`\~`\,\lowercase{\def~}{\discretionary{\hbox{\char`\,}}{\Wrappedafterbreak}{\hbox{\char`\,}}}% 
            \lccode`\~`\;\lowercase{\def~}{\discretionary{\hbox{\char`\;}}{\Wrappedafterbreak}{\hbox{\char`\;}}}% 
            \lccode`\~`\:\lowercase{\def~}{\discretionary{\hbox{\char`\:}}{\Wrappedafterbreak}{\hbox{\char`\:}}}% 
            \lccode`\~`\?\lowercase{\def~}{\discretionary{\hbox{\char`\?}}{\Wrappedafterbreak}{\hbox{\char`\?}}}% 
            \lccode`\~`\!\lowercase{\def~}{\discretionary{\hbox{\char`\!}}{\Wrappedafterbreak}{\hbox{\char`\!}}}% 
            \lccode`\~`\/\lowercase{\def~}{\discretionary{\hbox{\char`\/}}{\Wrappedafterbreak}{\hbox{\char`\/}}}% 
            \catcode`\.\active
            \catcode`\,\active 
            \catcode`\;\active
            \catcode`\:\active
            \catcode`\?\active
            \catcode`\!\active
            \catcode`\/\active 
            \lccode`\~`\~ 	
        }
    \makeatother

    \let\OriginalVerbatim=\Verbatim
    \makeatletter
    \renewcommand{\Verbatim}[1][1]{%
        %\parskip\z@skip
        \sbox\Wrappedcontinuationbox {\Wrappedcontinuationsymbol}%
        \sbox\Wrappedvisiblespacebox {\FV@SetupFont\Wrappedvisiblespace}%
        \def\FancyVerbFormatLine ##1{\hsize\linewidth
            \vtop{\raggedright\hyphenpenalty\z@\exhyphenpenalty\z@
                \doublehyphendemerits\z@\finalhyphendemerits\z@
                \strut ##1\strut}%
        }%
        % If the linebreak is at a space, the latter will be displayed as visible
        % space at end of first line, and a continuation symbol starts next line.
        % Stretch/shrink are however usually zero for typewriter font.
        \def\FV@Space {%
            \nobreak\hskip\z@ plus\fontdimen3\font minus\fontdimen4\font
            \discretionary{\copy\Wrappedvisiblespacebox}{\Wrappedafterbreak}
            {\kern\fontdimen2\font}%
        }%
        
        % Allow breaks at special characters using \PYG... macros.
        \Wrappedbreaksatspecials
        % Breaks at punctuation characters . , ; ? ! and / need catcode=\active 	
        \OriginalVerbatim[#1,codes*=\Wrappedbreaksatpunct]%
    }
    \makeatother

    % Exact colors from NB
    \definecolor{incolor}{HTML}{303F9F}
    \definecolor{outcolor}{HTML}{D84315}
    \definecolor{cellborder}{HTML}{CFCFCF}
    \definecolor{cellbackground}{HTML}{F7F7F7}
    
    % prompt
    \makeatletter
    \newcommand{\boxspacing}{\kern\kvtcb@left@rule\kern\kvtcb@boxsep}
    \makeatother
    \newcommand{\prompt}[4]{
        {\ttfamily\llap{{\color{#2}[#3]:\hspace{3pt}#4}}\vspace{-\baselineskip}}
    }
    

    
    % Prevent overflowing lines due to hard-to-break entities
    \sloppy 
    % Setup hyperref package
    \hypersetup{
      breaklinks=true,  % so long urls are correctly broken across lines
      colorlinks=true,
      urlcolor=urlcolor,
      linkcolor=linkcolor,
      citecolor=citecolor,
      }
    % Slightly bigger margins than the latex defaults
    
    \geometry{verbose,tmargin=1in,bmargin=1in,lmargin=1in,rmargin=1in}
    
    

\begin{document}
    
    \maketitle
    
    

    
    \hypertarget{stats-21---hw-3---ethan-warren}{%
\section{Stats 21 - HW 3 - Ethan
Warren}\label{stats-21---hw-3---ethan-warren}}

    Homework copyright Miles Chen. Problems have been adapted from teh
exercises in Think Python 2nd Ed by Allen B. Downey.

The questions have been entered into this document. You will modify the
document by entering your code.

Make sure you run the cell so the requested output is visible. Download
the finished document as a PDF file. If you are unable to convert it to
a PDF, you can download it as an HTML file and then print to PDF.

\textbf{Homework is an opportunity to practice coding and to practice
problem solving. Doing exercises is where you will do most of your
learning.}

\textbf{Copying someone else's solutions takes away your learning
opportunities. It is also academic dishonesty.}

    \hypertarget{reading}{%
\subsection{Reading}\label{reading}}

\begin{itemize}
\tightlist
\item
  Chapters 11, 12, and 14
\end{itemize}

Please keep up with the reading. The chapters are short.

    \hypertarget{exercise-11.1}{%
\subsubsection{Exercise 11.1}\label{exercise-11.1}}

Write a function that reads the words in \texttt{words.txt} and stores
them as keys in a dictionary. It doesn't matter what the values are.
Then you can use the in operator as a fast way to check whether a string
is in the dictionary.

Use the same \texttt{words.txt} file from HW2.

Do Exercise 10.10 but this time searching the dictionary using the
\texttt{in} operator. You can see how much faster it is to search a
dictionary.

    \begin{tcolorbox}[breakable, size=fbox, boxrule=1pt, pad at break*=1mm,colback=cellbackground, colframe=cellborder]
\prompt{In}{incolor}{1}{\boxspacing}
\begin{Verbatim}[commandchars=\\\{\}]
\PY{k}{def} \PY{n+nf}{make\PYZus{}word\PYZus{}dict}\PY{p}{(}\PY{p}{)}\PY{p}{:}
    \PY{k}{global} \PY{n}{word\PYZus{}dict} 
    \PY{n}{word\PYZus{}dict} \PY{o}{=} \PY{n+nb}{dict}\PY{p}{(}\PY{p}{)}
    \PY{n}{words} \PY{o}{=} \PY{n+nb}{open}\PY{p}{(}\PY{l+s+s2}{\PYZdq{}}\PY{l+s+s2}{words.txt}\PY{l+s+s2}{\PYZdq{}}\PY{p}{)}
    \PY{k}{for} \PY{n}{line} \PY{o+ow}{in} \PY{n}{words}\PY{p}{:}
        \PY{n}{word} \PY{o}{=} \PY{n}{line}\PY{o}{.}\PY{n}{strip}\PY{p}{(}\PY{p}{)}
        \PY{n}{word\PYZus{}dict}\PY{p}{[}\PY{n}{word}\PY{p}{]} \PY{o}{=} \PY{l+m+mi}{1}
\end{Verbatim}
\end{tcolorbox}

    \begin{tcolorbox}[breakable, size=fbox, boxrule=1pt, pad at break*=1mm,colback=cellbackground, colframe=cellborder]
\prompt{In}{incolor}{2}{\boxspacing}
\begin{Verbatim}[commandchars=\\\{\}]
\PY{n}{make\PYZus{}word\PYZus{}dict}\PY{p}{(}\PY{p}{)}
\end{Verbatim}
\end{tcolorbox}

    \begin{tcolorbox}[breakable, size=fbox, boxrule=1pt, pad at break*=1mm,colback=cellbackground, colframe=cellborder]
\prompt{In}{incolor}{3}{\boxspacing}
\begin{Verbatim}[commandchars=\\\{\}]
\PY{l+s+s2}{\PYZdq{}}\PY{l+s+s2}{hello}\PY{l+s+s2}{\PYZdq{}} \PY{o+ow}{in} \PY{n}{word\PYZus{}dict}
\end{Verbatim}
\end{tcolorbox}

            \begin{tcolorbox}[breakable, size=fbox, boxrule=.5pt, pad at break*=1mm, opacityfill=0]
\prompt{Out}{outcolor}{3}{\boxspacing}
\begin{Verbatim}[commandchars=\\\{\}]
True
\end{Verbatim}
\end{tcolorbox}
        
    \hypertarget{exercise-11.4}{%
\subsubsection{Exercise 11.4}\label{exercise-11.4}}

In Exercise 10.7, you created a function called
\texttt{has\_duplicates()}. It takes a list as a parameter and returns
\texttt{True} if there is any object that appears more than once in the
list.

Use a dictionary to write a faster, simpler version of
\texttt{has\_duplicates()}.

    \begin{tcolorbox}[breakable, size=fbox, boxrule=1pt, pad at break*=1mm,colback=cellbackground, colframe=cellborder]
\prompt{In}{incolor}{4}{\boxspacing}
\begin{Verbatim}[commandchars=\\\{\}]
\PY{k}{def} \PY{n+nf}{has\PYZus{}duplicates}\PY{p}{(}\PY{n}{t}\PY{p}{)}\PY{p}{:}
    \PY{n}{d} \PY{o}{=} \PY{n+nb}{dict}\PY{p}{(}\PY{p}{)}
    \PY{k}{for} \PY{n}{x} \PY{o+ow}{in} \PY{n}{t}\PY{p}{:}
        \PY{k}{if} \PY{n}{x} \PY{o+ow}{in} \PY{n}{d}\PY{p}{:}
            \PY{k}{return} \PY{k+kc}{True}
        \PY{k}{else}\PY{p}{:}
            \PY{n}{d}\PY{p}{[}\PY{n}{x}\PY{p}{]} \PY{o}{=} \PY{l+m+mi}{1}
    \PY{k}{return} \PY{k+kc}{False}
\end{Verbatim}
\end{tcolorbox}

    \begin{tcolorbox}[breakable, size=fbox, boxrule=1pt, pad at break*=1mm,colback=cellbackground, colframe=cellborder]
\prompt{In}{incolor}{5}{\boxspacing}
\begin{Verbatim}[commandchars=\\\{\}]
\PY{n}{has\PYZus{}duplicates}\PY{p}{(}\PY{p}{[}\PY{l+s+s1}{\PYZsq{}}\PY{l+s+s1}{a}\PY{l+s+s1}{\PYZsq{}}\PY{p}{,}\PY{l+s+s1}{\PYZsq{}}\PY{l+s+s1}{b}\PY{l+s+s1}{\PYZsq{}}\PY{p}{,}\PY{l+s+s1}{\PYZsq{}}\PY{l+s+s1}{b}\PY{l+s+s1}{\PYZsq{}}\PY{p}{,}\PY{l+s+s1}{\PYZsq{}}\PY{l+s+s1}{c}\PY{l+s+s1}{\PYZsq{}}\PY{p}{]}\PY{p}{)}
\end{Verbatim}
\end{tcolorbox}

            \begin{tcolorbox}[breakable, size=fbox, boxrule=.5pt, pad at break*=1mm, opacityfill=0]
\prompt{Out}{outcolor}{5}{\boxspacing}
\begin{Verbatim}[commandchars=\\\{\}]
True
\end{Verbatim}
\end{tcolorbox}
        
    \begin{tcolorbox}[breakable, size=fbox, boxrule=1pt, pad at break*=1mm,colback=cellbackground, colframe=cellborder]
\prompt{In}{incolor}{6}{\boxspacing}
\begin{Verbatim}[commandchars=\\\{\}]
\PY{n}{has\PYZus{}duplicates}\PY{p}{(}\PY{p}{[}\PY{l+s+s1}{\PYZsq{}}\PY{l+s+s1}{a}\PY{l+s+s1}{\PYZsq{}}\PY{p}{,}\PY{l+s+s1}{\PYZsq{}}\PY{l+s+s1}{b}\PY{l+s+s1}{\PYZsq{}}\PY{p}{,}\PY{l+s+s1}{\PYZsq{}}\PY{l+s+s1}{c}\PY{l+s+s1}{\PYZsq{}}\PY{p}{,}\PY{l+s+s1}{\PYZsq{}}\PY{l+s+s1}{a}\PY{l+s+s1}{\PYZsq{}}\PY{p}{]}\PY{p}{)}
\end{Verbatim}
\end{tcolorbox}

            \begin{tcolorbox}[breakable, size=fbox, boxrule=.5pt, pad at break*=1mm, opacityfill=0]
\prompt{Out}{outcolor}{6}{\boxspacing}
\begin{Verbatim}[commandchars=\\\{\}]
True
\end{Verbatim}
\end{tcolorbox}
        ### Exercise 11.5

A Caesar cipher is a weak form of encryption that involves 'rotating' each letter by a fixed number of places. To rotate a letter means to shift it through the alphabet, wrapping around to the beginning if necessary. "A" rotated by 3 is "D". "Z" rotated by 1 is "A".

Two words are "rotate pairs" if you can rotate one of them and get the other. For example, "cheer" rotated by 7 is "jolly".

Write a script that reads in the wordlist `words.txt` and finds all the rotate pairs of words that are 5 letters or longer.

One function that could be helpful is the function `ord()` which converts a character to a numeric code. Keep in mind that numeric codes for uppercase and lowercase letters are different.

Some hints:

- you can write helper functions, such as a function that will rotate a letter by a certain number and/or another function that will rotate a word by a number of letters
- to keep your script running quickly, you should use the wordlist dictionary from exercise 11.1
    \begin{tcolorbox}[breakable, size=fbox, boxrule=1pt, pad at break*=1mm,colback=cellbackground, colframe=cellborder]
\prompt{In}{incolor}{7}{\boxspacing}
\begin{Verbatim}[commandchars=\\\{\}]
\PY{k}{def} \PY{n+nf}{rotate\PYZus{}letter}\PY{p}{(}\PY{n}{letter}\PY{p}{,} \PY{n}{by}\PY{p}{)}\PY{p}{:}
    \PY{n}{u} \PY{o}{=} \PY{n+nb}{ord}\PY{p}{(}\PY{n}{letter}\PY{p}{)}
    \PY{n}{newu} \PY{o}{=} \PY{n}{u} \PY{o}{+} \PY{n}{by}
    \PY{k}{if} \PY{n}{newu} \PY{o}{\PYZgt{}} \PY{l+m+mi}{122}\PY{p}{:}
        \PY{n}{newu} \PY{o}{=} \PY{n}{newu} \PY{o}{\PYZhy{}} \PY{l+m+mi}{26}
    \PY{n}{new\PYZus{}letter} \PY{o}{=} \PY{n+nb}{chr}\PY{p}{(}\PY{n}{newu}\PY{p}{)}
    \PY{k}{return} \PY{n}{new\PYZus{}letter}
\end{Verbatim}
\end{tcolorbox}

    \begin{tcolorbox}[breakable, size=fbox, boxrule=1pt, pad at break*=1mm,colback=cellbackground, colframe=cellborder]
\prompt{In}{incolor}{8}{\boxspacing}
\begin{Verbatim}[commandchars=\\\{\}]
\PY{k}{def} \PY{n+nf}{rotate\PYZus{}word}\PY{p}{(}\PY{n}{word}\PY{p}{,} \PY{n}{by}\PY{p}{)}\PY{p}{:}
    \PY{n}{new\PYZus{}word} \PY{o}{=} \PY{l+s+s2}{\PYZdq{}}\PY{l+s+s2}{\PYZdq{}}
    \PY{k}{for} \PY{n}{char} \PY{o+ow}{in} \PY{n}{word}\PY{p}{:}
        \PY{n}{new\PYZus{}word} \PY{o}{=} \PY{n}{new\PYZus{}word} \PY{o}{+} \PY{n}{rotate\PYZus{}letter}\PY{p}{(}\PY{n}{char}\PY{p}{,} \PY{n}{by}\PY{p}{)}
    \PY{k}{return} \PY{n}{new\PYZus{}word}
\end{Verbatim}
\end{tcolorbox}

    \begin{tcolorbox}[breakable, size=fbox, boxrule=1pt, pad at break*=1mm,colback=cellbackground, colframe=cellborder]
\prompt{In}{incolor}{9}{\boxspacing}
\begin{Verbatim}[commandchars=\\\{\}]
\PY{n}{track\PYZus{}dict} \PY{o}{=} \PY{n+nb}{dict}\PY{p}{(}\PY{p}{)}
\PY{k}{for} \PY{n}{word} \PY{o+ow}{in} \PY{n}{word\PYZus{}dict}\PY{p}{:}
    \PY{k}{if} \PY{n+nb}{len}\PY{p}{(}\PY{n}{word}\PY{p}{)} \PY{o}{\PYZlt{}} \PY{l+m+mi}{5}\PY{p}{:}
        \PY{k}{continue}
    \PY{k}{if} \PY{n}{word} \PY{o+ow}{in} \PY{n}{track\PYZus{}dict}\PY{p}{:}
        \PY{k}{continue}
    \PY{n}{l} \PY{o}{=} \PY{p}{[}\PY{n}{word}\PY{p}{]}
    \PY{k}{for} \PY{n}{i} \PY{o+ow}{in} \PY{n+nb}{range}\PY{p}{(}\PY{l+m+mi}{1}\PY{p}{,} \PY{l+m+mi}{26}\PY{p}{)}\PY{p}{:}
        \PY{n}{new\PYZus{}string} \PY{o}{=} \PY{n}{rotate\PYZus{}word}\PY{p}{(}\PY{n}{word}\PY{p}{,} \PY{n}{i}\PY{p}{)}
        \PY{k}{if} \PY{n}{new\PYZus{}string} \PY{o+ow}{in} \PY{n}{word\PYZus{}dict}\PY{p}{:}
            \PY{n}{l}\PY{o}{.}\PY{n}{append}\PY{p}{(}\PY{n}{new\PYZus{}string}\PY{p}{)}
            \PY{n}{l}\PY{o}{.}\PY{n}{append}\PY{p}{(}\PY{n}{i}\PY{p}{)}
            \PY{n}{track\PYZus{}dict}\PY{p}{[}\PY{n}{word}\PY{p}{]} \PY{o}{=} \PY{l+m+mi}{1}
            \PY{n}{track\PYZus{}dict}\PY{p}{[}\PY{n}{new\PYZus{}string}\PY{p}{]} \PY{o}{=} \PY{l+m+mi}{1}
    \PY{k}{if} \PY{n+nb}{len}\PY{p}{(}\PY{n}{l}\PY{p}{)} \PY{o}{\PYZgt{}} \PY{l+m+mi}{1}\PY{p}{:}
        \PY{n+nb}{print}\PY{p}{(}\PY{n}{l}\PY{p}{)}
\end{Verbatim}
\end{tcolorbox}

    \begin{Verbatim}[commandchars=\\\{\}]
['abjurer', 'nowhere', 13]
['adder', 'beefs', 1]
['ahull', 'gnarr', 6]
['alkyd', 'epoch', 4]
['alula', 'tenet', 19]
['ambit', 'setal', 18]
['anteed', 'bouffe', 1]
['aptly', 'timer', 19]
['arena', 'river', 17]
['baiza', 'tsars', 18]
['banjo', 'ferns', 4]
['benni', 'ruddy', 16]
['biffs', 'holly', 6]
['bolls', 'hurry', 6]
['bombyx', 'hushed', 6]
['bourg', 'viola', 20]
['buffi', 'hallo', 6]
['bulls', 'harry', 6]
['bunny', 'sleep', 17]
['butyl', 'hazer', 6]
['chain', 'ingot', 6]
['cheer', 'jolly', 7]
['clasp', 'raphe', 15]
['cogon', 'sewed', 16]
['commy', 'secco', 16]
['corky', 'wiles', 20]
['craal', 'penny', 13]
['credo', 'shute', 16]
['creel', 'perry', 13]
['cubed', 'melon', 10]
['curly', 'wolfs', 20]
['cushy', 'wombs', 20]
['danio', 'herms', 4]
['dated', 'spits', 15]
['dawted', 'splits', 15]
['dazed', 'spots', 15]
['didos', 'pupae', 12]
['dolls', 'wheel', 19]
['drips', 'octad', 11]
['ebony', 'uredo', 16]
['eches', 'kinky', 6]
['fadge', 'torus', 14]
['fagot', 'touch', 14]
['fills', 'lorry', 6]
['fizzy', 'kneed', 5]
['frena', 'wiver', 17]
['frere', 'serer', 13]
['fusion', 'layout', 6]
['ganja', 'kerne', 4]
['gassy', 'smeek', 12]
['ginny', 'motte', 6]
['gnarl', 'xeric', 17]
['golem', 'murks', 6]
['golly', 'murre', 6]
['green', 'terra', 13]
['gulfs', 'marly', 6]
['gulls', 'marry', 6]
['gummy', 'masse', 6]
['gunny', 'matte', 6]
['hints', 'styed', 11]
['hoggs', 'tasse', 12]
['hotel', 'ovals', 7]
['inkier', 'purply', 7]
['jerky', 'snath', 9]
['jiffs', 'polly', 6]
['jimmy', 'posse', 6]
['jinni', 'potto', 6]
['jinns', 'potty', 6]
['johns', 'punty', 6]
['lallan', 'pepper', 4]
['later', 'shaly', 7]
['latex', 'shale', 7]
['linum', 'rotas', 6]
['luffs', 'rally', 6]
['lutea', 'pyxie', 4]
['manful', 'thumbs', 7]
['mills', 'sorry', 6]
['mocha', 'suing', 6]
['molas', 'surgy', 6]
['muffs', 'sally', 6]
['mulch', 'sarin', 6]
['mumms', 'sassy', 6]
['munch', 'satin', 6]
['muumuu', 'weewee', 10]
['noggs', 'tummy', 6]
['nulls', 'tarry', 6]
['nutty', 'tazze', 6]
['oxter', 'vealy', 7]
['pecan', 'tiger', 4]
['primero', 'sulphur', 3]
['pulpy', 'varve', 6]
['ratan', 'vexer', 4]
['sheer', 'tiffs', 1]
['sneer', 'toffs', 1]
['steeds', 'tuffet', 1]
['steer', 'tuffs', 1]
['teloi', 'whorl', 3]
    \end{Verbatim}

    \begin{tcolorbox}[breakable, size=fbox, boxrule=1pt, pad at break*=1mm,colback=cellbackground, colframe=cellborder]
\prompt{In}{incolor}{10}{\boxspacing}
\begin{Verbatim}[commandchars=\\\{\}]
\PY{n}{has\PYZus{}duplicates}\PY{p}{(}\PY{p}{[}\PY{l+s+s1}{\PYZsq{}}\PY{l+s+s1}{a}\PY{l+s+s1}{\PYZsq{}}\PY{p}{,}\PY{l+s+s1}{\PYZsq{}}\PY{l+s+s1}{b}\PY{l+s+s1}{\PYZsq{}}\PY{p}{,}\PY{l+s+s1}{\PYZsq{}}\PY{l+s+s1}{b}\PY{l+s+s1}{\PYZsq{}}\PY{p}{,}\PY{l+s+s1}{\PYZsq{}}\PY{l+s+s1}{c}\PY{l+s+s1}{\PYZsq{}}\PY{p}{]}\PY{p}{)}
\end{Verbatim}
\end{tcolorbox}

            \begin{tcolorbox}[breakable, size=fbox, boxrule=.5pt, pad at break*=1mm, opacityfill=0]
\prompt{Out}{outcolor}{10}{\boxspacing}
\begin{Verbatim}[commandchars=\\\{\}]
True
\end{Verbatim}
\end{tcolorbox}
        
    \hypertarget{exercise-12.2}{%
\subsubsection{Exercise 12.2}\label{exercise-12.2}}

Write a program that reads a word list from a file (see Section 9.1) and
finds all the sets of words that are anagrams.

After finding all anagram sets, print the list of all anagram sets that
have 6 or more entries in it.

Here is an example of what the output might look like:

\begin{verbatim}
['abets', 'baste', 'bates', 'beast', 'beats', 'betas', 'tabes']
['acers', 'acres', 'cares', 'carse', 'escar', 'races', 'scare', 'serac']
['acred', 'arced', 'cadre', 'cared', 'cedar', 'raced']
...
\end{verbatim}

Hint:

First traverse the entire wordlist to build a dictionary that maps from
a collection of letters to a list of words that can be spelled with
those letters. The question is, how can you represent the collection of
letters in a way that can be used as a key? i.e.~The word ``eat'' and
the word ``tea'' should be in a list associated with a key
(`a',`e',`t').

    \begin{tcolorbox}[breakable, size=fbox, boxrule=1pt, pad at break*=1mm,colback=cellbackground, colframe=cellborder]
\prompt{In}{incolor}{11}{\boxspacing}
\begin{Verbatim}[commandchars=\\\{\}]
\PY{n}{fin} \PY{o}{=} \PY{n+nb}{open}\PY{p}{(}\PY{l+s+s2}{\PYZdq{}}\PY{l+s+s2}{words.txt}\PY{l+s+s2}{\PYZdq{}}\PY{p}{)}
\PY{n}{anagrams\PYZus{}dict} \PY{o}{=} \PY{n+nb}{dict}\PY{p}{(}\PY{p}{)}
\PY{k}{for} \PY{n}{line} \PY{o+ow}{in} \PY{n}{fin}\PY{p}{:}
    \PY{n}{word} \PY{o}{=} \PY{n}{line}\PY{o}{.}\PY{n}{strip}\PY{p}{(}\PY{p}{)}
    \PY{c+c1}{\PYZsh{} letters will become the key for the anagrams dictionary. We need it to be sorted and immutable, so we create a tuple.}
    \PY{n}{letters} \PY{o}{=} \PY{n+nb}{tuple}\PY{p}{(}\PY{n+nb}{sorted}\PY{p}{(}\PY{n+nb}{list}\PY{p}{(}\PY{n}{word}\PY{p}{)}\PY{p}{)}\PY{p}{)}
    \PY{k}{if} \PY{n}{letters} \PY{o+ow}{in} \PY{n}{anagrams\PYZus{}dict}\PY{p}{:}
        \PY{n}{anagrams\PYZus{}dict}\PY{p}{[}\PY{n}{letters}\PY{p}{]}\PY{o}{.}\PY{n}{append}\PY{p}{(}\PY{n}{word}\PY{p}{)}
    \PY{k}{else}\PY{p}{:}
        \PY{n}{anagrams\PYZus{}dict}\PY{p}{[}\PY{n}{letters}\PY{p}{]} \PY{o}{=} \PY{p}{[}\PY{n}{word}\PY{p}{]}
\PY{c+c1}{\PYZsh{} Make a list of the dictionary values that have more than 6 anagram words and output that.}
\PY{n}{out} \PY{o}{=} \PY{n+nb}{list}\PY{p}{(}\PY{p}{)}
\PY{k}{for} \PY{n}{l} \PY{o+ow}{in} \PY{n}{anagrams\PYZus{}dict}\PY{o}{.}\PY{n}{values}\PY{p}{(}\PY{p}{)}\PY{p}{:}
    \PY{k}{if} \PY{n+nb}{len}\PY{p}{(}\PY{n}{l}\PY{p}{)} \PY{o}{\PYZgt{}} \PY{l+m+mi}{5}\PY{p}{:}
        \PY{n}{out}\PY{o}{.}\PY{n}{append}\PY{p}{(}\PY{n}{l}\PY{p}{)}
\PY{n}{out}
\end{Verbatim}
\end{tcolorbox}

            \begin{tcolorbox}[breakable, size=fbox, boxrule=.5pt, pad at break*=1mm, opacityfill=0]
\prompt{Out}{outcolor}{11}{\boxspacing}
\begin{Verbatim}[commandchars=\\\{\}]
[['abets', 'baste', 'bates', 'beast', 'beats', 'betas', 'tabes'],
 ['acers', 'acres', 'cares', 'carse', 'escar', 'races', 'scare', 'serac'],
 ['acred', 'arced', 'cadre', 'cared', 'cedar', 'raced'],
 ['aiders', 'deairs', 'irades', 'raised', 'redias', 'resaid'],
 ['airts', 'astir', 'sitar', 'stair', 'stria', 'tarsi'],
 ['alert', 'alter', 'artel', 'later', 'ratel', 'taler'],
 ['alerting', 'altering', 'integral', 'relating', 'tanglier', 'triangle'],
 ['alerts',
  'alters',
  'artels',
  'estral',
  'laster',
  'ratels',
  'salter',
  'slater',
  'staler',
  'stelar',
  'talers'],
 ['alevin', 'alvine', 'valine', 'veinal', 'venial', 'vineal'],
 ['algins', 'aligns', 'lasing', 'liangs', 'ligans', 'lingas', 'signal'],
 ['aliens', 'alines', 'elains', 'lianes', 'saline', 'silane'],
 ['aligners', 'engrails', 'nargiles', 'realigns', 'signaler', 'slangier'],
 ['amens', 'manes', 'manse', 'means', 'mensa', 'names', 'nemas'],
 ['anestri',
  'nastier',
  'ratines',
  'retains',
  'retinas',
  'retsina',
  'stainer',
  'stearin'],
 ['angriest',
  'astringe',
  'ganister',
  'gantries',
  'granites',
  'ingrates',
  'rangiest'],
 ['apers',
  'asper',
  'pares',
  'parse',
  'pears',
  'prase',
  'presa',
  'rapes',
  'reaps',
  'spare',
  'spear'],
 ['ardeb', 'barde', 'bared', 'beard', 'bread', 'debar'],
 ['ardebs', 'bardes', 'beards', 'breads', 'debars', 'sabred', 'serdab'],
 ['ares', 'arse', 'ears', 'eras', 'rase', 'sear', 'sera'],
 ['aretes', 'easter', 'eaters', 'reseat', 'seater', 'teaser'],
 ['aridest', 'astride', 'diaster', 'disrate', 'staider', 'tardies', 'tirades'],
 ['aril', 'lair', 'liar', 'lira', 'rail', 'rial'],
 ['ariled', 'derail', 'dialer', 'laired', 'railed', 'redial', 'relaid'],
 ['arils', 'lairs', 'liars', 'liras', 'rails', 'rials'],
 ['arles', 'earls', 'lares', 'laser', 'lears', 'rales', 'reals', 'seral'],
 ['armets', 'master', 'maters', 'matres', 'ramets', 'stream', 'tamers'],
 ['arrest', 'rarest', 'raster', 'raters', 'starer', 'tarres', 'terras'],
 ['artiest', 'artiste', 'attires', 'iratest', 'ratites', 'striate', 'tastier'],
 ['ashed', 'deash', 'hades', 'heads', 'sadhe', 'shade'],
 ['aspen', 'napes', 'neaps', 'panes', 'peans', 'sneap', 'spean'],
 ['aspers',
  'parses',
  'passer',
  'prases',
  'repass',
  'spares',
  'sparse',
  'spears'],
 ['aspirer', 'parries', 'praiser', 'rapiers', 'raspier', 'repairs'],
 ['ates', 'east', 'eats', 'etas', 'sate', 'seat', 'seta', 'teas'],
 ['bares', 'baser', 'bears', 'braes', 'saber', 'sabre'],
 ['canter', 'centra', 'nectar', 'recant', 'tanrec', 'trance'],
 ['canters', 'nectars', 'recants', 'scanter', 'tanrecs', 'trances'],
 ['capers',
  'crapes',
  'escarp',
  'pacers',
  'parsec',
  'recaps',
  'scrape',
  'secpar',
  'spacer'],
 ['caress', 'carses', 'crases', 'escars', 'scares', 'seracs'],
 ['caret', 'carte', 'cater', 'crate', 'react', 'recta', 'trace'],
 ['carets',
  'cartes',
  'caster',
  'caters',
  'crates',
  'reacts',
  'recast',
  'traces'],
 ['carpels', 'clasper', 'parcels', 'placers', 'reclasp', 'scalper'],
 ['claroes', 'coalers', 'escolar', 'oracles', 'recoals', 'solacer'],
 ['corset', 'coster', 'escort', 'rectos', 'scoter', 'sector'],
 ['cruet', 'curet', 'cuter', 'eruct', 'recut', 'truce'],
 ['cruets', 'cruset', 'curets', 'eructs', 'rectus', 'recuts', 'truces'],
 ['dater', 'derat', 'rated', 'tared', 'trade', 'tread'],
 ['dearths', 'hardest', 'hardset', 'hatreds', 'threads', 'trashed'],
 ['deers', 'drees', 'redes', 'reeds', 'seder', 'sered'],
 ['deils', 'delis', 'idles', 'isled', 'sidle', 'slide'],
 ['deist', 'diets', 'dites', 'edits', 'sited', 'stied', 'tides'],
 ['deltas', 'desalt', 'lasted', 'salted', 'slated', 'staled'],
 ['deposit', 'dopiest', 'podites', 'posited', 'sopited', 'topside'],
 ['diols', 'idols', 'lidos', 'sloid', 'soldi', 'solid'],
 ['drapes', 'padres', 'parsed', 'rasped', 'spader', 'spared', 'spread'],
 ['earings',
  'erasing',
  'gainers',
  'reagins',
  'regains',
  'reginas',
  'searing',
  'seringa'],
 ['easters', 'reseats', 'searest', 'seaters', 'teasers', 'tessera'],
 ['easting', 'eatings', 'ingates', 'ingesta', 'seating', 'teasing'],
 ['elastin', 'entails', 'nailset', 'salient', 'saltine', 'tenails'],
 ['emits', 'items', 'metis', 'mites', 'smite', 'stime', 'times'],
 ['empires', 'emprise', 'epimers', 'imprese', 'premies', 'premise', 'spireme'],
 ['enosis', 'eosins', 'essoin', 'noesis', 'noises', 'ossein', 'sonsie'],
 ['enters', 'nester', 'rentes', 'resent', 'tenser', 'ternes'],
 ['esprit', 'priest', 'ripest', 'sprite', 'stripe', 'tripes'],
 ['esprits', 'persist', 'priests', 'spriest', 'sprites', 'stirpes', 'stripes'],
 ['ester', 'reest', 'reset', 'steer', 'stere', 'terse', 'trees'],
 ['esters', 'reests', 'resets', 'serest', 'steers', 'steres'],
 ['estrange', 'grantees', 'greatens', 'negaters', 'reagents', 'sergeant'],
 ['estrin',
  'inerts',
  'insert',
  'inters',
  'niters',
  'nitres',
  'sinter',
  'triens',
  'trines'],
 ['estrous', 'oestrus', 'ousters', 'sourest', 'souters', 'stoures', 'tussore'],
 ['eviler', 'levier', 'liever', 'relive', 'revile', 'veiler'],
 ['hales', 'heals', 'leash', 'selah', 'shale', 'sheal'],
 ['hassel', 'hassle', 'lashes', 'selahs', 'shales', 'sheals'],
 ['hectors', 'rochets', 'rotches', 'tochers', 'torches', 'troches'],
 ['lapse', 'leaps', 'pales', 'peals', 'pleas', 'salep', 'sepal', 'spale'],
 ['lavers', 'ravels', 'salver', 'serval', 'slaver', 'velars', 'versal'],
 ['laves', 'salve', 'slave', 'vales', 'valse', 'veals'],
 ['leapt', 'lepta', 'palet', 'petal', 'plate', 'pleat'],
 ['least',
  'setal',
  'slate',
  'stale',
  'steal',
  'stela',
  'taels',
  'tales',
  'teals',
  'tesla'],
 ['leasts', 'slates', 'stales', 'steals', 'tassel', 'teslas'],
 ['luster', 'lustre', 'result', 'rustle', 'sutler', 'ulster'],
 ['lusters', 'lustres', 'results', 'rustles', 'sutlers', 'ulsters'],
 ['mates', 'meats', 'satem', 'steam', 'tames', 'teams'],
 ['merits', 'mister', 'miters', 'mitres', 'remits', 'smiter', 'timers'],
 ['nestor', 'noters', 'stoner', 'tenors', 'tensor', 'toners', 'trones'],
 ['notes', 'onset', 'seton', 'steno', 'stone', 'tones'],
 ['opts', 'post', 'pots', 'spot', 'stop', 'tops'],
 ['painters', 'pantries', 'pertains', 'pinaster', 'pristane', 'repaints'],
 ['palest',
  'palets',
  'pastel',
  'petals',
  'plates',
  'pleats',
  'septal',
  'staple'],
 ['palters', 'persalt', 'plaster', 'platers', 'psalter', 'stapler'],
 ['parties', 'pastier', 'piaster', 'piastre', 'pirates', 'traipse'],
 ['parts', 'prats', 'sprat', 'strap', 'tarps', 'traps'],
 ['paste', 'pates', 'peats', 'septa', 'spate', 'tapes', 'tepas'],
 ['paster', 'paters', 'prates', 'repast', 'tapers', 'trapes'],
 ['peers', 'peres', 'perse', 'prees', 'prese', 'speer', 'spree'],
 ['peris', 'piers', 'pries', 'prise', 'ripes', 'speir', 'spier', 'spire'],
 ['petrous', 'posture', 'pouters', 'proteus', 'spouter', 'troupes'],
 ['piles', 'plies', 'slipe', 'speil', 'spiel', 'spile'],
 ['realist', 'retails', 'saltier', 'saltire', 'slatier', 'tailers'],
 ['recepts', 'respect', 'scepter', 'sceptre', 'specter', 'spectre'],
 ['reigns', 'renigs', 'resign', 'sering', 'signer', 'singer'],
 ['reins', 'resin', 'rinse', 'risen', 'serin', 'siren'],
 ['resaw', 'sawer', 'sewar', 'sware', 'swear', 'wares', 'wears'],
 ['staw', 'swat', 'taws', 'twas', 'wast', 'wats'],
 ['stow', 'swot', 'tows', 'twos', 'wost', 'wots']]
\end{Verbatim}
\end{tcolorbox}
        
    \hypertarget{numpy-exercises}{%
\subsubsection{NumPy Exercises}\label{numpy-exercises}}

For the following exercises, be sure to print the desired output. You
will not receive credit for problems that do not print the desired
output.

Some of these exercises will require the use of functions in numpy that
may not have been covered in class. Look up documentation on how to use
them. I always recommend checking the official reference at
https://numpy.org/doc/stable/reference/index.html

    \begin{tcolorbox}[breakable, size=fbox, boxrule=1pt, pad at break*=1mm,colback=cellbackground, colframe=cellborder]
\prompt{In}{incolor}{12}{\boxspacing}
\begin{Verbatim}[commandchars=\\\{\}]
\PY{k+kn}{import} \PY{n+nn}{numpy} \PY{k}{as} \PY{n+nn}{np}
\PY{n}{np}\PY{o}{.}\PY{n}{random}\PY{o}{.}\PY{n}{seed}\PY{p}{(}\PY{l+m+mi}{1}\PY{p}{)}
\end{Verbatim}
\end{tcolorbox}

    \hypertarget{exercise-np.1}{%
\paragraph{Exercise NP.1:}\label{exercise-np.1}}

Task: Create an array \texttt{b} of 10 random integers selected between
0-99

Desired output: \texttt{{[}37\ 12\ 72\ \ 9\ 75\ \ 5\ 79\ 64\ 16\ \ 1{]}}
of course yours might be different

    \begin{tcolorbox}[breakable, size=fbox, boxrule=1pt, pad at break*=1mm,colback=cellbackground, colframe=cellborder]
\prompt{In}{incolor}{13}{\boxspacing}
\begin{Verbatim}[commandchars=\\\{\}]
\PY{n}{b} \PY{o}{=} \PY{n}{np}\PY{o}{.}\PY{n}{random}\PY{o}{.}\PY{n}{randint}\PY{p}{(}\PY{l+m+mi}{0}\PY{p}{,} \PY{l+m+mi}{100}\PY{p}{,} \PY{l+m+mi}{10}\PY{p}{)}
\PY{n+nb}{print}\PY{p}{(}\PY{n}{b}\PY{p}{)}
\end{Verbatim}
\end{tcolorbox}

    \begin{Verbatim}[commandchars=\\\{\}]
[37 12 72  9 75  5 79 64 16  1]
    \end{Verbatim}

    \hypertarget{np.1b}{%
\paragraph{NP.1b:}\label{np.1b}}

Task: reverse the elements in \texttt{b}. Hint: Try slicing the array,
but backwards

Desired output: \texttt{{[}\ 1\ 16\ 64\ 79\ \ 5\ 75\ \ 9\ 72\ 12\ 37{]}}
yours will be different

    \begin{tcolorbox}[breakable, size=fbox, boxrule=1pt, pad at break*=1mm,colback=cellbackground, colframe=cellborder]
\prompt{In}{incolor}{14}{\boxspacing}
\begin{Verbatim}[commandchars=\\\{\}]
\PY{n+nb}{print}\PY{p}{(}\PY{n}{b}\PY{p}{[}\PY{p}{:}\PY{p}{:}\PY{o}{\PYZhy{}}\PY{l+m+mi}{1}\PY{p}{]}\PY{p}{)}
\end{Verbatim}
\end{tcolorbox}

    \begin{Verbatim}[commandchars=\\\{\}]
[ 1 16 64 79  5 75  9 72 12 37]
    \end{Verbatim}

    \hypertarget{np.2a}{%
\paragraph{NP.2a:}\label{np.2a}}

Task: Create an array \texttt{c} of 1000 random values selected from a
normal distribution centered at 100 with sd = 15, rounded to 1 decimal
place. Print only the first 10 values.

Desired output:
\texttt{{[}\ 92.1\ \ 83.9\ 113.\ \ \ 65.5\ 126.2\ \ 88.6\ 104.8\ \ 96.3\ 121.9\ \ 69.1{]}}
of course your values may be different

    \begin{tcolorbox}[breakable, size=fbox, boxrule=1pt, pad at break*=1mm,colback=cellbackground, colframe=cellborder]
\prompt{In}{incolor}{15}{\boxspacing}
\begin{Verbatim}[commandchars=\\\{\}]
\PY{n}{c} \PY{o}{=} \PY{n}{np}\PY{o}{.}\PY{n}{random}\PY{o}{.}\PY{n}{normal}\PY{p}{(}\PY{l+m+mi}{100}\PY{p}{,} \PY{l+m+mi}{15}\PY{p}{,} \PY{l+m+mi}{1000}\PY{p}{)}
\PY{n}{c} \PY{o}{=} \PY{n}{np}\PY{o}{.}\PY{n}{round}\PY{p}{(}\PY{n}{c}\PY{p}{,} \PY{l+m+mi}{1}\PY{p}{)}
\PY{n+nb}{print}\PY{p}{(}\PY{n}{c}\PY{p}{[}\PY{p}{:}\PY{l+m+mi}{10}\PY{p}{]}\PY{p}{)}
\end{Verbatim}
\end{tcolorbox}

    \begin{Verbatim}[commandchars=\\\{\}]
[ 92.1  83.9 113.   65.5 126.2  88.6 104.8  96.3 121.9  69.1]
    \end{Verbatim}

    \hypertarget{np.2b}{%
\paragraph{NP.2b:}\label{np.2b}}

Perform a Shapiro-Wilk test to see if the values in c come from a normal
distribution. Report the p-value and appropriate conclusion.

Look up \texttt{scipy.stats.shapiro} for usage and details.

https://docs.scipy.org/doc/scipy/reference/generated/scipy.stats.shapiro.html

    \begin{tcolorbox}[breakable, size=fbox, boxrule=1pt, pad at break*=1mm,colback=cellbackground, colframe=cellborder]
\prompt{In}{incolor}{44}{\boxspacing}
\begin{Verbatim}[commandchars=\\\{\}]
\PY{k+kn}{from} \PY{n+nn}{scipy} \PY{k+kn}{import} \PY{n}{stats}
\PY{n}{shapiro\PYZus{}test} \PY{o}{=} \PY{n}{stats}\PY{o}{.}\PY{n}{shapiro}\PY{p}{(}\PY{n}{c}\PY{p}{)}
\PY{n}{out} \PY{o}{=} \PY{p}{(}\PY{l+s+s2}{\PYZdq{}}\PY{l+s+s2}{The p\PYZhy{}value is }\PY{l+s+si}{\PYZob{}\PYZcb{}}\PY{l+s+s2}{ which means that we cannot reject the null hypothesis. }\PY{l+s+s2}{\PYZdq{}} \PY{o}{+}
       \PY{l+s+s2}{\PYZdq{}}\PY{l+s+s2}{The values in c may follow a normal distribution.}\PY{l+s+s2}{\PYZdq{}}\PY{p}{)}
\PY{n+nb}{print}\PY{p}{(}\PY{n}{out}\PY{o}{.}\PY{n}{format}\PY{p}{(}\PY{n+nb}{round}\PY{p}{(}\PY{n}{shapiro\PYZus{}test}\PY{p}{[}\PY{l+m+mi}{1}\PY{p}{]}\PY{p}{,} \PY{l+m+mi}{3}\PY{p}{)}\PY{p}{)}\PY{p}{)}
\end{Verbatim}
\end{tcolorbox}

    \begin{Verbatim}[commandchars=\\\{\}]
The p-value is 0.179 which means that we cannot reject the null hypothesis. The
values in c may follow a normal distribution.
    \end{Verbatim}

    \hypertarget{np.2c}{%
\paragraph{NP.2c:}\label{np.2c}}

Identify and print the values in \texttt{c} that are more than 3
standard deviations from the mean of \texttt{c}. Report the proportion
of values that are more than 3 sd from the mean.

Desired output: \texttt{{[}\ 54.1\ ...\ 148.3\ {]}}

\texttt{0.32\ of\ the\ values\ are\ beyond\ 3\ sd\ from\ the\ mean.}

    \begin{tcolorbox}[breakable, size=fbox, boxrule=1pt, pad at break*=1mm,colback=cellbackground, colframe=cellborder]
\prompt{In}{incolor}{17}{\boxspacing}
\begin{Verbatim}[commandchars=\\\{\}]
\PY{n}{l} \PY{o}{=} \PY{n+nb}{list}\PY{p}{(}\PY{p}{)}
\PY{k}{for} \PY{n}{value} \PY{o+ow}{in} \PY{n}{c}\PY{p}{:}
    \PY{k}{if} \PY{n}{value} \PY{o}{\PYZlt{}} \PY{l+m+mi}{55} \PY{o+ow}{or} \PY{n}{value} \PY{o}{\PYZgt{}} \PY{l+m+mi}{145}\PY{p}{:}
        \PY{n}{l}\PY{o}{.}\PY{n}{append}\PY{p}{(}\PY{n}{value}\PY{p}{)}
\PY{n+nb}{print}\PY{p}{(}\PY{n}{np}\PY{o}{.}\PY{n}{array}\PY{p}{(}\PY{n}{l}\PY{p}{)}\PY{p}{)}
\PY{n+nb}{print}\PY{p}{(}\PY{l+s+s2}{\PYZdq{}}\PY{l+s+si}{\PYZob{}\PYZcb{}}\PY{l+s+si}{\PYZpc{} o}\PY{l+s+s2}{f the values are beyond 3 standard deviations from the mean}\PY{l+s+s2}{\PYZdq{}}\PY{o}{.}\PY{n}{format}\PY{p}{(}\PY{n+nb}{len}\PY{p}{(}\PY{n}{l}\PY{p}{)}\PY{o}{/}\PY{l+m+mi}{10}\PY{p}{)}\PY{p}{)}
\end{Verbatim}
\end{tcolorbox}

    \begin{Verbatim}[commandchars=\\\{\}]
[145.5 159.4 149.8  54.2]
0.4\% of the values are beyond 3 standard deviations from the mean
    \end{Verbatim}

    \hypertarget{np.3}{%
\paragraph{NP.3:}\label{np.3}}

Task: Make a 3x3 identity matrix called \texttt{I3}

Desired output:

\begin{verbatim}
[[1. 0. 0.]
 [0. 1. 0.]
 [0. 0. 1.]]
\end{verbatim}

    \begin{tcolorbox}[breakable, size=fbox, boxrule=1pt, pad at break*=1mm,colback=cellbackground, colframe=cellborder]
\prompt{In}{incolor}{18}{\boxspacing}
\begin{Verbatim}[commandchars=\\\{\}]
\PY{n}{I3} \PY{o}{=} \PY{n}{np}\PY{o}{.}\PY{n}{identity}\PY{p}{(}\PY{l+m+mi}{3}\PY{p}{)}
\PY{n+nb}{print}\PY{p}{(}\PY{n}{I3}\PY{p}{)}
\end{Verbatim}
\end{tcolorbox}

    \begin{Verbatim}[commandchars=\\\{\}]
[[1. 0. 0.]
 [0. 1. 0.]
 [0. 0. 1.]]
    \end{Verbatim}

    \hypertarget{np.4a}{%
\paragraph{NP.4a:}\label{np.4a}}

Task: Make a 10x10 array of values 1 to 100. Call it X

Desired output:

\begin{verbatim}
[[  1   2   3   4   5   6   7   8   9  10]
 [ 11  12  13  14  15  16  17  18  19  20]
 [ 21  22  23  24  25  26  27  28  29  30]
 [ 31  32  33  34  35  36  37  38  39  40]
 [ 41  42  43  44  45  46  47  48  49  50]
 [ 51  52  53  54  55  56  57  58  59  60]
 [ 61  62  63  64  65  66  67  68  69  70]
 [ 71  72  73  74  75  76  77  78  79  80]
 [ 81  82  83  84  85  86  87  88  89  90]
 [ 91  92  93  94  95  96  97  98  99 100]]
\end{verbatim}

    \begin{tcolorbox}[breakable, size=fbox, boxrule=1pt, pad at break*=1mm,colback=cellbackground, colframe=cellborder]
\prompt{In}{incolor}{19}{\boxspacing}
\begin{Verbatim}[commandchars=\\\{\}]
\PY{n}{X} \PY{o}{=} \PY{n}{np}\PY{o}{.}\PY{n}{arange}\PY{p}{(}\PY{l+m+mi}{1}\PY{p}{,} \PY{l+m+mi}{101}\PY{p}{,} \PY{l+m+mi}{1}\PY{p}{)}\PY{o}{.}\PY{n}{reshape}\PY{p}{(}\PY{p}{(}\PY{l+m+mi}{10}\PY{p}{,}\PY{l+m+mi}{10}\PY{p}{)}\PY{p}{)}
\PY{n+nb}{print}\PY{p}{(}\PY{n}{X}\PY{p}{)}
\end{Verbatim}
\end{tcolorbox}

    \begin{Verbatim}[commandchars=\\\{\}]
[[  1   2   3   4   5   6   7   8   9  10]
 [ 11  12  13  14  15  16  17  18  19  20]
 [ 21  22  23  24  25  26  27  28  29  30]
 [ 31  32  33  34  35  36  37  38  39  40]
 [ 41  42  43  44  45  46  47  48  49  50]
 [ 51  52  53  54  55  56  57  58  59  60]
 [ 61  62  63  64  65  66  67  68  69  70]
 [ 71  72  73  74  75  76  77  78  79  80]
 [ 81  82  83  84  85  86  87  88  89  90]
 [ 91  92  93  94  95  96  97  98  99 100]]
    \end{Verbatim}

    \hypertarget{np.4b}{%
\paragraph{NP.4b:}\label{np.4b}}

Task: Make a copy of X, call it Y (1 line). Replace all values in Y that
are not squares with 0 (1 line). see \texttt{numpy.isin()}

Desired output:

\begin{verbatim}
[[  1   0   0   4   0   0   0   0   9   0]
 [  0   0   0   0   0  16   0   0   0   0]
 [  0   0   0   0  25   0   0   0   0   0]
...
 [  0   0   0   0   0   0   0   0   0 100]]
\end{verbatim}

    \begin{tcolorbox}[breakable, size=fbox, boxrule=1pt, pad at break*=1mm,colback=cellbackground, colframe=cellborder]
\prompt{In}{incolor}{20}{\boxspacing}
\begin{Verbatim}[commandchars=\\\{\}]
\PY{n}{Y} \PY{o}{=} \PY{n}{X}
\PY{n}{Y}\PY{p}{[}\PY{n}{np}\PY{o}{.}\PY{n}{isin}\PY{p}{(}\PY{n}{X}\PY{p}{,} \PY{n}{np}\PY{o}{.}\PY{n}{arange}\PY{p}{(}\PY{l+m+mi}{1}\PY{p}{,} \PY{l+m+mi}{11}\PY{p}{)}\PY{o}{*}\PY{o}{*}\PY{l+m+mi}{2}\PY{p}{,} \PY{n}{invert}\PY{o}{=} \PY{k+kc}{True}\PY{p}{)}\PY{p}{]} \PY{o}{=} \PY{l+m+mi}{0}
\PY{n+nb}{print}\PY{p}{(}\PY{n}{Y}\PY{p}{)}
\end{Verbatim}
\end{tcolorbox}

    \begin{Verbatim}[commandchars=\\\{\}]
[[  1   0   0   4   0   0   0   0   9   0]
 [  0   0   0   0   0  16   0   0   0   0]
 [  0   0   0   0  25   0   0   0   0   0]
 [  0   0   0   0   0  36   0   0   0   0]
 [  0   0   0   0   0   0   0   0  49   0]
 [  0   0   0   0   0   0   0   0   0   0]
 [  0   0   0  64   0   0   0   0   0   0]
 [  0   0   0   0   0   0   0   0   0   0]
 [ 81   0   0   0   0   0   0   0   0   0]
 [  0   0   0   0   0   0   0   0   0 100]]
    \end{Verbatim}

    \hypertarget{np.5}{%
\paragraph{NP.5:}\label{np.5}}

Task: Use \texttt{np.tile()} to tile a 2x2 diagonal matrix of integers
to make a checkerboard pattern. Call the matrix \texttt{checkers}

Desired output:

\begin{verbatim}
[[1 0 1 0 1 0 1 0]
 [0 1 0 1 0 1 0 1]
 ...
 [1 0 1 0 1 0 1 0]
 [0 1 0 1 0 1 0 1]]
\end{verbatim}

    \begin{tcolorbox}[breakable, size=fbox, boxrule=1pt, pad at break*=1mm,colback=cellbackground, colframe=cellborder]
\prompt{In}{incolor}{21}{\boxspacing}
\begin{Verbatim}[commandchars=\\\{\}]
\PY{n}{checkers} \PY{o}{=} \PY{n}{np}\PY{o}{.}\PY{n}{tile}\PY{p}{(}\PY{n}{np}\PY{o}{.}\PY{n}{identity}\PY{p}{(}\PY{l+m+mi}{2}\PY{p}{)}\PY{o}{.}\PY{n}{astype}\PY{p}{(}\PY{n+nb}{int}\PY{p}{)}\PY{p}{,} \PY{p}{(}\PY{l+m+mi}{4}\PY{p}{,}\PY{l+m+mi}{4}\PY{p}{)}\PY{p}{)}
\PY{n+nb}{print}\PY{p}{(}\PY{n}{checkers}\PY{p}{)}
\end{Verbatim}
\end{tcolorbox}

    \begin{Verbatim}[commandchars=\\\{\}]
[[1 0 1 0 1 0 1 0]
 [0 1 0 1 0 1 0 1]
 [1 0 1 0 1 0 1 0]
 [0 1 0 1 0 1 0 1]
 [1 0 1 0 1 0 1 0]
 [0 1 0 1 0 1 0 1]
 [1 0 1 0 1 0 1 0]
 [0 1 0 1 0 1 0 1]]
    \end{Verbatim}

    \hypertarget{np.6}{%
\paragraph{NP.6:}\label{np.6}}

Task: convert the values in \texttt{f\_temp} from Farenheit to celsius.
The conversion is subtract 32, then multiply by 5/9. Round to 1 decimal
place.

Desired output:

\begin{verbatim}
[[21.1 21.7 22.2 22.8 23.3 23.9]
 ...
 [34.4 35.  35.6 36.1 36.7 37.2]]
\end{verbatim}

    \begin{tcolorbox}[breakable, size=fbox, boxrule=1pt, pad at break*=1mm,colback=cellbackground, colframe=cellborder]
\prompt{In}{incolor}{22}{\boxspacing}
\begin{Verbatim}[commandchars=\\\{\}]
\PY{c+c1}{\PYZsh{} do not modify}
\PY{n}{f\PYZus{}temp}  \PY{o}{=} \PY{n}{np}\PY{o}{.}\PY{n}{arange}\PY{p}{(}\PY{l+m+mi}{70}\PY{p}{,}\PY{l+m+mi}{100}\PY{p}{)}\PY{o}{.}\PY{n}{reshape}\PY{p}{(}\PY{p}{(}\PY{l+m+mi}{5}\PY{p}{,}\PY{l+m+mi}{6}\PY{p}{)}\PY{p}{)}
\PY{n+nb}{print}\PY{p}{(}\PY{n}{np}\PY{o}{.}\PY{n}{round}\PY{p}{(}\PY{p}{(}\PY{n}{f\PYZus{}temp} \PY{o}{\PYZhy{}} \PY{l+m+mi}{32}\PY{p}{)}\PY{o}{*}\PY{l+m+mi}{5}\PY{o}{/}\PY{l+m+mi}{9}\PY{p}{,} \PY{l+m+mi}{1}\PY{p}{)}\PY{p}{)}
\end{Verbatim}
\end{tcolorbox}

    \begin{Verbatim}[commandchars=\\\{\}]
[[21.1 21.7 22.2 22.8 23.3 23.9]
 [24.4 25.  25.6 26.1 26.7 27.2]
 [27.8 28.3 28.9 29.4 30.  30.6]
 [31.1 31.7 32.2 32.8 33.3 33.9]
 [34.4 35.  35.6 36.1 36.7 37.2]]
    \end{Verbatim}

    \hypertarget{np.7}{%
\paragraph{NP.7:}\label{np.7}}

Task: Convert values in the matrix \texttt{x} into z-scores by column,
call it matrix \texttt{z}. That is: each column should have a mean of 0
and std of 1. (subtract the column mean, and divide by the column std).
(not required, but see if you can do this in one line)

Print the column means and column std to show that they have been
standardized.

Desired output:

\begin{verbatim}
[[-1.09996745 -0.47901666  0.8816739 ]
 [ 0.9495002   1.18844641  0.11324252]
...
 [-0.60705751 -1.08536687 -0.57430135]
 [-1.28156585 -0.81250928  1.52877401]]
 
 [ 6.66133815e-17  1.11022302e-16 -1.11022302e-16]
[1. 1. 1.]
\end{verbatim}

    \begin{tcolorbox}[breakable, size=fbox, boxrule=1pt, pad at break*=1mm,colback=cellbackground, colframe=cellborder]
\prompt{In}{incolor}{23}{\boxspacing}
\begin{Verbatim}[commandchars=\\\{\}]
\PY{c+c1}{\PYZsh{} do not modify}
\PY{n}{np}\PY{o}{.}\PY{n}{random}\PY{o}{.}\PY{n}{seed}\PY{p}{(}\PY{l+m+mi}{100}\PY{p}{)}
\PY{n}{x} \PY{o}{=} \PY{n}{np}\PY{o}{.}\PY{n}{random}\PY{o}{.}\PY{n}{randint}\PY{p}{(}\PY{l+m+mi}{100}\PY{p}{,}\PY{n}{size} \PY{o}{=} \PY{l+m+mi}{30}\PY{p}{)}\PY{o}{.}\PY{n}{reshape}\PY{p}{(}\PY{l+m+mi}{10}\PY{p}{,}\PY{l+m+mi}{3}\PY{p}{)}
\end{Verbatim}
\end{tcolorbox}

    \begin{tcolorbox}[breakable, size=fbox, boxrule=1pt, pad at break*=1mm,colback=cellbackground, colframe=cellborder]
\prompt{In}{incolor}{24}{\boxspacing}
\begin{Verbatim}[commandchars=\\\{\}]
\PY{n}{z} \PY{o}{=} \PY{n}{np}\PY{o}{.}\PY{n}{array}\PY{p}{(}\PY{p}{[}\PY{p}{(}\PY{n}{x}\PY{p}{[}\PY{p}{:}\PY{p}{,}\PY{n}{j}\PY{p}{]} \PY{o}{\PYZhy{}} \PY{n}{np}\PY{o}{.}\PY{n}{mean}\PY{p}{(}\PY{n}{x}\PY{p}{[}\PY{p}{:}\PY{p}{,}\PY{n}{j}\PY{p}{]}\PY{p}{)}\PY{p}{)} \PY{o}{/} \PY{n}{np}\PY{o}{.}\PY{n}{std}\PY{p}{(}\PY{n}{x}\PY{p}{[}\PY{p}{:}\PY{p}{,}\PY{n}{j}\PY{p}{]}\PY{p}{)} \PY{k}{for} \PY{n}{j} \PY{o+ow}{in} \PY{n+nb}{range}\PY{p}{(}\PY{l+m+mi}{3}\PY{p}{)}\PY{p}{]}\PY{p}{)}\PY{o}{.}\PY{n}{T}
\PY{n+nb}{print}\PY{p}{(}\PY{n}{z}\PY{p}{)}
\end{Verbatim}
\end{tcolorbox}

    \begin{Verbatim}[commandchars=\\\{\}]
[[-1.09996745 -0.47901666  0.8816739 ]
 [ 0.9495002   1.18844641  0.11324252]
 [-1.04808219  1.64320907  0.27501755]
 [ 1.23486912  0.40019114  0.84123014]
 [ 1.23486912 -0.78219177 -0.45297008]
 [-0.68488539 -0.75187426  0.5985676 ]
 [ 0.19716398 -0.72155675 -1.46406399]
 [ 1.10515598  1.40066898 -1.74717029]
 [-0.60705751 -1.08536687 -0.57430135]
 [-1.28156585 -0.81250928  1.52877401]]
    \end{Verbatim}

    \begin{tcolorbox}[breakable, size=fbox, boxrule=1pt, pad at break*=1mm,colback=cellbackground, colframe=cellborder]
\prompt{In}{incolor}{25}{\boxspacing}
\begin{Verbatim}[commandchars=\\\{\}]
\PY{n+nb}{print}\PY{p}{(}\PY{p}{[}\PY{n}{np}\PY{o}{.}\PY{n}{mean}\PY{p}{(}\PY{n}{z}\PY{p}{[}\PY{p}{:}\PY{p}{,}\PY{n}{j}\PY{p}{]}\PY{p}{)} \PY{k}{for} \PY{n}{j} \PY{o+ow}{in} \PY{n+nb}{range}\PY{p}{(}\PY{l+m+mi}{3}\PY{p}{)}\PY{p}{]}\PY{p}{)}
\PY{n+nb}{print}\PY{p}{(}\PY{p}{[}\PY{n}{np}\PY{o}{.}\PY{n}{std}\PY{p}{(}\PY{n}{z}\PY{p}{[}\PY{p}{:}\PY{p}{,}\PY{n}{j}\PY{p}{]}\PY{p}{)} \PY{k}{for} \PY{n}{j} \PY{o+ow}{in} \PY{n+nb}{range}\PY{p}{(}\PY{l+m+mi}{3}\PY{p}{)}\PY{p}{]}\PY{p}{)}
\end{Verbatim}
\end{tcolorbox}

    \begin{Verbatim}[commandchars=\\\{\}]
[4.4408920985006264e-17, 6.661338147750939e-17, -1.3322676295501878e-16]
[1.0000000000000002, 1.0, 0.9999999999999999]
    \end{Verbatim}

    \hypertarget{np.8}{%
\paragraph{NP.8:}\label{np.8}}

Task: Convert values in the matrix \texttt{x} into scaled values from 0
to 10. That is take each column and scale values linearly so that the
largest value is 10, and the smallest value in the column is 0. Round
results to 2 decimal places. Call the result \texttt{y}

(Not required, but see if you can do the calculations in one line.)

Desired output:

\begin{verbatim}
[[ 0.72  2.22  8.02]
 [ 8.87  8.33  5.68]
...
 [ 2.68  0.    3.58]
 [ 0.    1.   10.  ]]
\end{verbatim}

    \begin{tcolorbox}[breakable, size=fbox, boxrule=1pt, pad at break*=1mm,colback=cellbackground, colframe=cellborder]
\prompt{In}{incolor}{26}{\boxspacing}
\begin{Verbatim}[commandchars=\\\{\}]
\PY{c+c1}{\PYZsh{} do not modify}
\PY{n}{np}\PY{o}{.}\PY{n}{random}\PY{o}{.}\PY{n}{seed}\PY{p}{(}\PY{l+m+mi}{100}\PY{p}{)}
\PY{n}{x} \PY{o}{=} \PY{n}{np}\PY{o}{.}\PY{n}{random}\PY{o}{.}\PY{n}{randint}\PY{p}{(}\PY{l+m+mi}{100}\PY{p}{,}\PY{n}{size} \PY{o}{=} \PY{l+m+mi}{30}\PY{p}{)}\PY{o}{.}\PY{n}{reshape}\PY{p}{(}\PY{l+m+mi}{10}\PY{p}{,}\PY{l+m+mi}{3}\PY{p}{)}
\end{Verbatim}
\end{tcolorbox}

    \begin{tcolorbox}[breakable, size=fbox, boxrule=1pt, pad at break*=1mm,colback=cellbackground, colframe=cellborder]
\prompt{In}{incolor}{41}{\boxspacing}
\begin{Verbatim}[commandchars=\\\{\}]
\PY{n}{y} \PY{o}{=} \PY{n}{np}\PY{o}{.}\PY{n}{array}\PY{p}{(}\PY{p}{[}\PY{l+m+mi}{10}\PY{o}{/}\PY{n}{np}\PY{o}{.}\PY{n}{max}\PY{p}{(}\PY{n}{x}\PY{p}{[}\PY{p}{:}\PY{p}{,}\PY{n}{j}\PY{p}{]}\PY{o}{\PYZhy{}}\PY{n}{np}\PY{o}{.}\PY{n}{min}\PY{p}{(}\PY{n}{x}\PY{p}{[}\PY{p}{:}\PY{p}{,}\PY{n}{j}\PY{p}{]}\PY{p}{)}\PY{p}{)} \PY{o}{*} \PY{p}{(}\PY{n}{x}\PY{p}{[}\PY{p}{:}\PY{p}{,}\PY{n}{j}\PY{p}{]} \PY{o}{\PYZhy{}} \PY{n}{np}\PY{o}{.}\PY{n}{min}\PY{p}{(}\PY{n}{x}\PY{p}{[}\PY{p}{:}\PY{p}{,}\PY{n}{j}\PY{p}{]}\PY{p}{)}\PY{p}{)} \PY{k}{for} \PY{n}{j} \PY{o+ow}{in} \PY{n+nb}{range}\PY{p}{(}\PY{l+m+mi}{3}\PY{p}{)}\PY{p}{]}\PY{p}{)}\PY{o}{.}\PY{n}{T}\PY{o}{.}\PY{n}{round}\PY{p}{(}\PY{l+m+mi}{2}\PY{p}{)}
\PY{n+nb}{print}\PY{p}{(}\PY{n}{y}\PY{p}{)}
\end{Verbatim}
\end{tcolorbox}

    \begin{Verbatim}[commandchars=\\\{\}]
[[ 0.72  2.22  8.02]
 [ 8.87  8.33  5.68]
 [ 0.93 10.    6.17]
 [10.    5.44  7.9 ]
 [10.    1.11  3.95]
 [ 2.37  1.22  7.16]
 [ 5.88  1.33  0.86]
 [ 9.48  9.11  0.  ]
 [ 2.68  0.    3.58]
 [ 0.    1.   10.  ]]
    \end{Verbatim}

    \hypertarget{np.9}{%
\paragraph{NP.9:}\label{np.9}}

Task: Replace all NaN values in the matrix \texttt{x} with 0.

Desired output:

\begin{verbatim}
[[ 8. 24. 67.]
 [87. 79.  0.]
 [10.  0. 52.]
 [ 0. 53. 66.]]
\end{verbatim}

    \begin{tcolorbox}[breakable, size=fbox, boxrule=1pt, pad at break*=1mm,colback=cellbackground, colframe=cellborder]
\prompt{In}{incolor}{28}{\boxspacing}
\begin{Verbatim}[commandchars=\\\{\}]
\PY{c+c1}{\PYZsh{} do not modify}
\PY{n}{np}\PY{o}{.}\PY{n}{random}\PY{o}{.}\PY{n}{seed}\PY{p}{(}\PY{l+m+mi}{100}\PY{p}{)}
\PY{n}{x} \PY{o}{=} \PY{n}{np}\PY{o}{.}\PY{n}{random}\PY{o}{.}\PY{n}{randint}\PY{p}{(}\PY{l+m+mi}{100}\PY{p}{,}\PY{n}{size} \PY{o}{=} \PY{l+m+mi}{12}\PY{p}{)}\PY{o}{.}\PY{n}{reshape}\PY{p}{(}\PY{l+m+mi}{4}\PY{p}{,}\PY{l+m+mi}{3}\PY{p}{)}\PY{o}{.}\PY{n}{astype}\PY{p}{(}\PY{l+s+s1}{\PYZsq{}}\PY{l+s+s1}{float}\PY{l+s+s1}{\PYZsq{}}\PY{p}{)}
\PY{n}{row} \PY{o}{=} \PY{n}{np}\PY{o}{.}\PY{n}{array}\PY{p}{(}\PY{p}{[}\PY{l+m+mi}{1}\PY{p}{,} \PY{l+m+mi}{2}\PY{p}{,} \PY{l+m+mi}{3}\PY{p}{]}\PY{p}{)}
\PY{n}{col} \PY{o}{=} \PY{n}{np}\PY{o}{.}\PY{n}{array}\PY{p}{(}\PY{p}{[}\PY{l+m+mi}{2}\PY{p}{,} \PY{l+m+mi}{1}\PY{p}{,} \PY{l+m+mi}{0}\PY{p}{]}\PY{p}{)}
\PY{n}{x}\PY{p}{[}\PY{n}{row}\PY{p}{,} \PY{n}{col}\PY{p}{]} \PY{o}{=} \PY{n}{np}\PY{o}{.}\PY{n}{nan}
\end{Verbatim}
\end{tcolorbox}

    \begin{tcolorbox}[breakable, size=fbox, boxrule=1pt, pad at break*=1mm,colback=cellbackground, colframe=cellborder]
\prompt{In}{incolor}{29}{\boxspacing}
\begin{Verbatim}[commandchars=\\\{\}]
\PY{n}{np}\PY{o}{.}\PY{n}{nan\PYZus{}to\PYZus{}num}\PY{p}{(}\PY{n}{x}\PY{p}{)}
\end{Verbatim}
\end{tcolorbox}

            \begin{tcolorbox}[breakable, size=fbox, boxrule=.5pt, pad at break*=1mm, opacityfill=0]
\prompt{Out}{outcolor}{29}{\boxspacing}
\begin{Verbatim}[commandchars=\\\{\}]
array([[ 8., 24., 67.],
       [87., 79.,  0.],
       [10.,  0., 52.],
       [ 0., 53., 66.]])
\end{Verbatim}
\end{tcolorbox}
        

    % Add a bibliography block to the postdoc
    
    
    
\end{document}
